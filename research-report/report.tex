%
% Author:  Stepan Pressl
% Last Edit Date: 19.3.2023
%
\documentclass[twoside,11pt]{article}
\usepackage[a4paper]{geometry}
\geometry{verbose,tmargin=2.5cm,bmargin=2cm,lmargin=2cm,rmargin=2cm}
\usepackage{fancyhdr}
\pagestyle{fancy}

\usepackage{lmodern}
\usepackage[T1]{fontenc}
\usepackage[utf8]{inputenc}
\usepackage[czech]{babel}
\usepackage{amsmath}
\usepackage{icomma}
\usepackage{url}
\usepackage{multirow}
\usepackage{subcaption}
\usepackage{graphicx}
\usepackage{epstopdf}
\usepackage[unicode=true, bookmarks=true,bookmarksnumbered=true,
bookmarksopen=false, breaklinks=false,pdfborder={0 0 0},
pdfpagemode=UseNone,backref=false,colorlinks=true] {hyperref}
\usepackage{xkeyval} % Inline todonotes
\usepackage[textsize = footnotesize]{todonotes}
\presetkeys{todonotes}{inline}{}

\newcommand{\pd}{\partial}
\newcommand{\pdfrac}[2]{\frac{\pd #1}{\pd #2}}
\newcommand{\jim}{\mathrm{j}}
\newcommand{\jd}[2]{#1\,\mathrm{#2}}
\newcommand{\limgt}[2]{\lim_{#1 \rightarrow #2}} % lim #1 goes to #2
\newcommand{\mrm}[1]{\mathrm{#1}}
\newcommand{\ohm}{\Omega}
\newcommand{\degc}{^\circ\mathrm{C}}
\newcommand{\cdotn}{\!\cdot\!}

% smaz aktualni page layout
\fancyhf{}
% zahlavi
\usepackage{titling}
\fancyhf[HC]{\thetitle}
\fancyhf[HLE,HRO]{\theauthor}
\fancyhf[HRE,HLO]{\today}
 %zapati
\fancyhf[FLE,FRO]{\thepage}

\title{ABBCCC - Report k projektu Ostrovní Energie}
\author{Vibašštěto Team}
\date{\today}

\begin{document}
\maketitle
Tento dokument popisuje metologii použitou při řešení úlohy Ostrovní Energie.
Na rozdíl od konkrétního zadání s použitím AWE/PEM elektrolyzérů používáme
obecný model elektrolyzérů. Používáme i obecný model takzvaných bufferů,
které slouží na uložení nadbytečně vyprodukované energie.

Cílem je ze zadaných obecných hodnot pomocí optimalizačních metod vypočítat
optimální strategie řízení energie. Při konkrétním návrhu finálního řešení
lze použít konkrétní AWE/PEM elektrolyzéry a buffery, mezi které patří
například akumulátory elektrické energie či gravitační baterie.

Finální grafická aplikace dovoluje uživateli zadat konkrétní hodnoty,
které budou popsány v následující 
použitých elektrolyzérů

\section{Popis modelu}

\end{document}















